\documentclass{report}
\usepackage[style = numeric, backend = biber]{biblatex}
\usepackage{graphicx, paralist, listings, fancybox, geometry, caption, floatrow, xcolor, color, colortbl, fancyhdr, amsmath, mathtools}
\usepackage{fouriernc}
\usepackage[T1]{fontenc}
\geometry{a4paper, top = 15mm, left = 5mm, right = 5mm, bottom = 17mm}
\fancypagestyle{plain}{
\fancyhead[L]{\texttt{GAME-DA} documentation}
\fancyhead[R]{\textsc{\texttt{GAME-DA} development team}}
\fancyfoot[C]{\thepage}
\addtolength\footskip{12pt}}
\definecolor{table_green}{rgb}{0, 0.6, 0}
\title{\texttt{GAME-DA} Documentation}
\author{\texttt{GAME-DA} Development Team}
\date{}
\newcommand{\md}[1]{\frac{D#1}{Dt}}
\newcommand{\omegabi}{\text{{\osgbi ω}}}
\newcommand{\mubi}{\text{{\osgbi μ}}}
\newcommand{\sigmabi}{\text{{\osgbi σ}}}
\newcommand{\epsilonbi}{\text{{\osgbi ϵ}}}
\newcommand{\etabi}{\text{{\osgbi η}}}
\newcommand{\zetabi}{\text{{\osgbi ζ}}}
\addbibresource{references.bib}
\DeclareFieldFormat[article]{title}{{#1}}

\begin{document}

\maketitle

\tableofcontents

\chapter{Introduction}
\label{sec:introduction}

\texttt{GAME-DA} provides \textit{data assimilation (DA)} funcitonality for the \texttt{GAME} model. The theoretical derivations are presented in \cite{kompendium}. This documentation contains the technical details of the implementations of the data assimilation algorithms. Three different methods of DA are implemented in GAME-DA:
%
\begin{itemize}
\item optimum interplation (OI)
\item three-dimensional variational data assimilation (3D-Var)
\item four-dimensional variational data assimilation (4D-Var)
\end{itemize}
%
Depending on the resolution and the type and amount of observations, certain methods are more suitable than others.

In any of the three cases, the basic workflow is as follows:
%
\begin{enumerate}
\item downloading observations
\item bringing the observations into a standardized format
\item executing the data assimilation itself, resulting in an input file for the model run
\end{enumerate}

\chapter{Optimum interpolation}
\label{chap:optimum_interpolation}

Let $M \geq 1$ be the number of degrees of freedom of the model, $N \geq 1$ the number of observations, $\mathbf{h} \in \mathbb{R}^N$ the observations as reconstructed from the background state, $\mathbf{y} \in \mathbb{R}^N$ the actual observations, $\overleftrightarrow{H} \in \mathbb{R}^{N\times M}$ the Jacobian of the observations operator, $\overleftrightarrow{B} \in \mathbb{R}^{M\times M}$ the background error covariance matrix and $\overleftrightarrow{R} \in \mathbb{R}^{N\times N}$ the observations error covariance matrix. $\overleftrightarrow{R}$ is usually assumed to be diagonal, which is an excellent approximation if the observation erros are of statistical nature. This does not lead to useful simplifications, however.

It is
%
\begin{align}
\left(\overleftrightarrow{H}\overleftrightarrow{B}\right)_{i, j} &= \sum_{k = 1}^{M}\overleftrightarrow{H}_{i, k}\overleftrightarrow{B}_{k, j}.
\end{align}
%
For $\overleftrightarrow{H}\overleftrightarrow{B}\overleftrightarrow{H}^T$, one obtains
%
\begin{align}
\left(\overleftrightarrow{H}\overleftrightarrow{B}\overleftrightarrow{H}^T\right)_{i, j} &= \left[\left(\overleftrightarrow{H}\overleftrightarrow{B}\right)\overleftrightarrow{H}^T\right]_{i, j} = \sum_{k = 1}^{M}\left(\overleftrightarrow{H}\overleftrightarrow{B}\right)_{i, k}\overleftrightarrow{H}^T_{k, j} = \sum_{k = 1}^{M}\left(\overleftrightarrow{H}\overleftrightarrow{B}\right)_{i, k}\overleftrightarrow{H}_{j, k} = \sum_{k = 1}^{M}\left(\sum_{l = 1}^{M}\overleftrightarrow{H}_{i, l}\overleftrightarrow{B}_{l, k}\right)\overleftrightarrow{H}_{j, k} = \sum_{k, l = 1}^M\overleftrightarrow{H}_{i, l}\overleftrightarrow{B}_{l, k}\overleftrightarrow{H}_{j, k}.
\end{align}
%
For the final result, we obtain
%
\begin{align}
x_i &= x_{B, i} + \sum_{j = 1}^N\left[\overleftrightarrow{B}\overleftrightarrow{H}^T\left(\overleftrightarrow{H}\overleftrightarrow{B}\overleftrightarrow{H}^T + \overleftrightarrow{R}\right)^{-1}\right]_{i, j}\left(y_j - h_j\right) = x_{B, i} + \sum_{j = 1}^N\left[\sum_{k = 1}^N\left(\overleftrightarrow{B}\overleftrightarrow{H}^T\right)_{i, k}\left(\overleftrightarrow{H}\overleftrightarrow{B}\overleftrightarrow{H}^T + \overleftrightarrow{R}\right)^{-1}_{k, j}\right]\left(y_j - h_j\right)\nonumber\\
&= x_{B, i} + \sum_{j = 1}^N\left[\sum_{k = 1}^N\left(\overleftrightarrow{H}\overleftrightarrow{B}\overleftrightarrow{H}^T + \overleftrightarrow{R}\right)^{-1}_{k, j}\left(\overleftrightarrow{B}\overleftrightarrow{H}^T\right)_{i, k}\right]\left(y_j - h_j\right) = x_{B, i} + \sum_{j = 1}^N\left[\sum_{k = 1}^N\left(\overleftrightarrow{H}\overleftrightarrow{B}\overleftrightarrow{H}^T + \overleftrightarrow{R}\right)^{-1}_{k, j}\left(\sum_{l = 1}^M\overleftrightarrow{B}_{i, l}\overleftrightarrow{H}^T_{l, k}\right)\right]\left(y_j - h_j\right)\nonumber
\end{align}
\begin{center}
\doublebox{\parbox{0.8\textwidth}{
\begin{center}
\begin{align}
\Rightarrow x_i &= x_{B, i} + \sum_{j = 1}^N\left[\sum_{k = 1}^N\left(\overleftrightarrow{H}\overleftrightarrow{B}\overleftrightarrow{H}^T + \overleftrightarrow{R}\right)^{-1}_{k, j}\left(\sum_{l = 1}^M\overleftrightarrow{B}_{i, l}\overleftrightarrow{H}_{k, l}\right)\right]\left(y_j - h_j\right).
\end{align}
\end{center}
}}
\end{center}

\section{Inclusion of moisture}
\label{sec:inclusion_of_moisture}

So far, moisture is taken into account in a separate assimilation process for efficiency.

\chapter{3D-Var}
\label{chap:3d-var}

\section{Technicalities}
\label{sec:technicalities_3d-var}

The observations used come from a time window $\left[-\frac{T}{2}, \frac{T}{2}\right]$ around the analysis time and are all taken to be valid at the analysis time. $T = 3$ h is a typical value..

\chapter{4D-Var}
\label{chap:4d-var}
%

\section{Technicalities}
\label{sec:technicalities_4d-var}

The observations used come from a time window $\left[-\frac{T}{2}, \frac{T}{2}\right]$ around the analysis time and are taken to be valid at individual time steps $n\Delta t$. $T = 6$ h and $\Delta t = 15$ min are typical values.

\appendix

\printbibliography

\end{document}













