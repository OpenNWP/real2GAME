\documentclass{report}
\usepackage[style = numeric, backend = biber]{biblatex}
\usepackage{graphicx, paralist, listings, fancybox, geometry, caption, floatrow, xcolor, color, colortbl, fancyhdr, amsmath, mathtools}
\usepackage{fouriernc}
\usepackage[T1]{fontenc}
\geometry{a4paper, top = 15mm, left = 5mm, right = 5mm, bottom = 17mm}
\fancypagestyle{plain}{
\fancyhead[L]{\texttt{ndvar} documentation}
\fancyhead[R]{\textsc{\texttt{ndvar} development team}}
\fancyfoot[C]{\thepage}
\addtolength\footskip{12pt}}
\definecolor{table_green}{rgb}{0, 0.6, 0}
\title{\texttt{ndvar} Documentation}
\author{\texttt{ndvar} Development Team}
\date{}
\newcommand{\md}[1]{\frac{D#1}{Dt}}
\newcommand{\omegabi}{\text{{\osgbi ω}}}
\newcommand{\mubi}{\text{{\osgbi μ}}}
\newcommand{\sigmabi}{\text{{\osgbi σ}}}
\newcommand{\epsilonbi}{\text{{\osgbi ϵ}}}
\newcommand{\etabi}{\text{{\osgbi η}}}
\newcommand{\zetabi}{\text{{\osgbi ζ}}}
\addbibresource{/home/max/my_texts/references.bib}
\DeclareFieldFormat[article]{title}{{#1}}

\begin{document}

\maketitle

\chapter{Optimum interpolation}
\label{chap:optimum_interpolation}

\begin{align}
\left(\overleftrightarrow{H}\overleftrightarrow{B}\right)_{i, j} &= \sum_{k = 1}^{M}\overleftrightarrow{H}_{i, k}\overleftrightarrow{B}_{k, j}.
\end{align}
%
In the special case, where $\overleftrightarrow{B}$ has diagonal form, this reduces to
%
\begin{align}
\left(\overleftrightarrow{H}\overleftrightarrow{B}\right)_{i, j} &= \sum_{k = 1}^{M}\overleftrightarrow{H}_{i, k}\overleftrightarrow{B}_{k, j}\delta_{k, j} = \sum_{k = 1}^{M}\overleftrightarrow{H}_{i, k}\overleftrightarrow{B}_{j, j}\delta_{k, j} = \overleftrightarrow{H}_{i, j}\overleftrightarrow{B}_{j, j}.
\end{align}
%
For $\overleftrightarrow{H}\overleftrightarrow{B}\overleftrightarrow{H}^T$, one obtains
%
\begin{align}
\left(\overleftrightarrow{H}\overleftrightarrow{B}\overleftrightarrow{H}^T\right)_{i, j} &= \left[\left(\overleftrightarrow{H}\overleftrightarrow{B}\right)\overleftrightarrow{H}^T\right]_{i, j} = \sum_{k = 1}^{M}\left(\overleftrightarrow{H}\overleftrightarrow{B}\right)_{i, k}\overleftrightarrow{H}^T_{k, j} = \sum_{k = 1}^{M}\left(\overleftrightarrow{H}\overleftrightarrow{B}\right)_{i, k}\overleftrightarrow{H}_{j, k} = \sum_{k = 1}^{M}\left(\sum_{l = 1}^{M}\overleftrightarrow{H}_{i, l}\overleftrightarrow{B}_{l, k}\right)\overleftrightarrow{H}_{j, k} = \sum_{k, l = 1}^M\overleftrightarrow{H}_{i, l}\overleftrightarrow{B}_{l, k}\overleftrightarrow{H}_{j, k}.
\end{align}
%
In the special case, where $\overleftrightarrow{B}$ has diagonal form, this reduces to
%
\begin{align}
\left(\overleftrightarrow{H}\overleftrightarrow{B}\overleftrightarrow{H}^T\right)_{i, j} &= \sum_{k = 1}^M\overleftrightarrow{H}_{i, k}\overleftrightarrow{B}_{k, k}\overleftrightarrow{H}_{j, k} = \sum_{k = 1}^M\overleftrightarrow{B}_{k, k}\left(\overleftrightarrow{H}_{i, k}\overleftrightarrow{H}_{j, k}\right).
\end{align}

\chapter{3D-Var}
\label{chap:3d-var}

\section{Technicalities}
\label{sec:technicalities_3d-var}

The observations used come from a time window $\left[-\frac{T}{2}, \frac{T}{2}\right]$ around the analysis time and are all taken to be valid at the analysis time. $T = 3$ h is a typical value..

\chapter{4D-Var}
\label{chap:4d-var}
%

\section{Technicalities}
\label{sec:technicalities_4d-var}

The observations used come from a time window $\left[-\frac{T}{2}, \frac{T}{2}\right]$ around the analysis time and are taken to be valid at individual time steps $n\Delta t$. $T = 6$ h and $\Delta t = 15$ min are typical values.

\appendix

\printbibliography

\end{document}













